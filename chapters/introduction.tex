%!TEX root = ../main.tex
% What is the project about? 
% What problem are you tackling? 
% What is your research question? 
% Why do these problems need solutions? Why are they important?
% What is the background to the problem? Who is the client? What do they want?
% What existing methods have been tried? How has I.T. been applied so far? 
% What constraints do you have? (Time, PCs, money, users, software etc) 
% What is the scope of what you have set yourself to do. What is not included?
% What broad approach was taken? (Summarise your broad approach the project) 
%%

\chapter{Introduction}
\label{chap:intro}

Cite paper as \cite{Haley:SecurityRequirementsEngineering}

\section{Background}
Compositing is one of many VFX techniques but can only be achieved if all the previous steps of the pipeline have been properly completed. The project will, not only, plan and develop for an implementation of 2D and 3D digitally created assets, but also go through the whole process of film development, from deciding on its style, planning the shots, recording, build the digital assets and then compositing them together. A survey at the end will ask questions to participants and try to understand and conclude what the opinions are when comparing these assets, in fixed camera shots, and get a brief overview on how long they think these took to create.

\section{Aims and Objectives}
The aims and objectives of the project were to mainly understand how the VFX pipeline works, what needs to be done before getting to Compositing, use 2D and 3D software to build digital assets and to compare how long they took to make, level of skill necessary and later conclude on which would have benefits for beginners and professional settings alike, in different situations. The project will only focus on fixed position shots which leaves a wide spectrum of other situations open to research.

\section{Structure}
\indent The project will be divided into five different sections: Literature Review, Design and Planning, Implementation, Testing, and Evaluation. 

\indent The Literature Review, also called Research, will address the information that was available priorly to the start of the study and use it as a starting point. The documentation will touch on the History of VFX, the most common types of VFX, where compositing appears in the pipeline, its techniques and lastly, mention Chroma Keying techniques and technical language used.

\indent As a result, the Design and Planning, will start by acknowledging how a film pre-production is done and use its inheritance to plan and decide how the clips will take place, look like and what visual effects will be done to accomplish the artists vision. The Design will have Mood Boards, Storyboards, the whole planning of the shooting days, a brief analysis on the equipment available and which will be used. It will also, decide on what software will be used for the post-production side of things and numerate the various requirements set for the project.

\indent Implementation is the longest section of the project; however, it has information on how all the steps were taken, and mentions, in detail, how everything was done, allowing for recreation by anyone that wants to. During the Production section, the setting up of the shots, will be considered and mentioned in the best way possible. While, on the Post-Production subsection, a variety of techniques and software use will be documented following the plans set out earlier.

\indent Testing is a brief section, where there will be an explanation about what the questionnaire was set out to find out, analyse and argue about the results and take conclusions of the findings.

\indent Lastly, the Evaluation section will have a reflection on the project, mention what kind of mistakes were made and if the project was to start over, what would have been done differently. Also, a conclusion about the project will be made here discussing everything and then set out paths that can be taken for future work if the opportunity materialises.

\section{Legal, Social, Ethical, Security and Professional Issues}
A project like this will always have legal, social, ethical, security and professional that must be considered, even, before starting as to diminish the probability of having a problem later that can put it out of commission. Having to work on so many steps of the pipeline, there was a need to plan for every step and document any of these, possible, issues and overview how they could be negated. To make a film, or in this case, make a couple of loose shots, it is necessary to ask permission to everyone that participates in it, directly or indirectly, like location owners.
While this could be a problem for other productions, it was set that it would be a one man’s crew and the filming location would be on my house’s private perimeter as a way of not having to pursue legal documentation that would hinder the project flow. When creating the storyboards, the number one rule was to not create any obscenity, blasphemy or show any deformation, illegality that could offend anyone in case they were to be shown to the public in general. Normally, for any film, a script would be necessary, but because these were a couple of shots without any real story there was no need to worry about that. While inspiration was taken from other films, no footage was used during the implementation. The lack of any music or sound effects released pressure on that front and the lack of product props would not create any bad brand interpretation.
During Production, security for the actors and crew alike, was the top priority. While there was one shot where the actor had to fake a fall, there were no rough edges around him that could hurt and even if he did happen to fall, the grass would let him fall in, overall, security.
In post-production, the use of creative commons license images and textures, was the main priority. The meshes were from my authorship, or in case of the ‘blender-osm’ used data that is available for free on the internet.
Issues like these arise all the time, due to lack of due diligence when producing a project like this. Before starting the production, it was necessary to plan and document legal, social, ethical, security and professional issues that could be forewarned beforehand. A good code of conduct is to ask for permission on everything that is used during the project that is not the author’s own work.
To finalise, the participants in the testing were asked to participate at their own discretion and could drop out at any moment. For those that completed it, there was, at no time, any information that could be directed back to them, therefore making the survey a hundred per cent anonymous to comply with GDPR requirements.
